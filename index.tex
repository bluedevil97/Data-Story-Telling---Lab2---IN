
\documentclass[preprint,12pt]{elsarticle}

\usepackage[spanish]{babel}
\usepackage{amssymb}
\usepackage{graphicx}
\usepackage{lineno}
\usepackage[utf8]{inputenc}
\usepackage{url}



\begin{document}
	
	\begin{frontmatter}
		
		
		\title{\huge Data Storytelling}
		
		\author{Mamani Ayala, Brandon        (2015052715)}
		\author{Quispe Mamani, Angelo	      (2015052826)}
		\author{Vizcarra Llanque, Jhordy	      (2015052719)}
		\author{Ordoñez Quilli, Ronald          (2015052821)}
		\author{Rodriguez Mamani, Juan      (2017057862)}
		
		\address{Tacna, Perú}
		
		\begin{abstract}
			%% Text of abstract
			texto
	
		\end{abstract}
\end{frontmatter}
%%

	
	%%
	%\linenumbers
	
	%% main text
	\section{Resumen}
		Aqui el resumen
\\

	%%
	
	%%
	%\linenumbers
	
	%% main text

\section{Objetivos}
		\begin{itemize}
		\item Objetivo 1: 
	\end{itemize}

	%%
	
	%%
	%\linenumbers
	
	%% main text

\section{Marco Teorico}
	
\subsection{Elementos clave}	
	El llamado Data Storytelling no es más que un enfoque estructurado sobre cómo comunicamos insights a partir de los datos, e involucra una combinación de 	tres elementos: datos, visualización y narrativa.
	\begin{center}
	\includegraphics[width=15cm]{./Imagenes/imagen1} 
	\end{center}
\begin{itemize}
\item	Narrativa + Datos = podremos explicar qué ha pasado y por qué un insight puede ser importante. Necesitaremos contexto para entender las conclusiones por completo.
\item Visualización + Datos = Enlighten. Cuando añadimos una visualización a nuestros datos, podemos iluminar a nuestra audiencia con insights que no habrían visto de otra manera.
\item	Narrativa + Visualización = Engagement. La combinación perfecta para lograr ese interés e incluso para entretener a nuestra audiencia. 
\end{itemize} 

	Pero, cuando unimos Visualización + Narración + Datos = Change, logramos contar una historia con nuestros datos, logramos influenciar y llevar a ese 			cambio que estábamos buscando.

	\begin{center}
	\includegraphics[width=10cm]{./Imagenes/imagen2} 
	\end{center}
Y es que la pasión por los datos, ¡tiene que ir acompañada también por la pasión de contar historias! Si comunicas pobremente tus insights o si llegas a conclusiones erróneas, puede ser prácticamente peor que no utilizar ningún dato.

\subsection{¿Por qué Data Storytelling?}
Si no tienes claro por qué incluir el elemento “historia” en tus análisis o no estás seguro de que vaya a funcionar, estos cuatro puntos te ayudarán a cambiar de opinión:

\begin{itemize}
\item	1.	Las historias son herramientas efectivas para transmitir la experiencia humana: esto ha sido así desde el inicio de los tiempos, pero ahora utilizamos datos y análisis para crear versiones mejoradas de esas historias. Gracias a ellas simplificamos y damos sentido a un mundo complejo.
\item 2.	Para inspirar el cambio, necesitamos que entiendan nuestra historia: no importa cuántas horas hay detrás de nuestro análisis, no lograremos nada si no nos logramos explicar ya sea con una narrativa o con gráficos pero, necesitamos una historia.
\item 3.	Las personas quieren evidencia del análisis que hay detrás: aunque nuestra audiencia no entienda el detalle de la analítica, sí quieren la evidencia de que hay datos detrás, ya que estas historias son más convincentes que solo una experiencia personal.
\item	4.	Contar en una breve historia el resultado de horas de trabajo: se necesitan presentaciones cortas, con ideas concretas adaptadas a los stakeholders que recibirán la información para hacer llegar tu mensaje de una manera simple. \\
	\begin{center}
	\includegraphics[width=10cm]{./Imagenes/imagen3} 
	\end{center}
\end{itemize} 


\subsection{Data Visualization}
	Es muy común que Data Storytelling se entienda solo como visualización de datos y, aunque como estamos viendo, es mucho más que eso, es cierto que la visualización es una parte esencial y muy potente como complementaria al análisis, para poder condensar grandes conjuntos de datos en una sola foto.

\subsection{Pero, ¿qué nos permite?}
	
\begin{itemize}
\item	1.	Comprensión rápida de la información: gracias a las representaciones gráficas podemos ver grandes cantidades de datos de forma clara y coherente, lo que facilita la extracción de conclusiones e insights. Ganaremos tiempo y eficiencia para solucionar problemas.
\item 2.	Identificar y actuar rápido sobre tendencias emergentes: incluso los archivos de datos casi infinitos, empiezan a tener sentido al representarse gráficamente; lo que nos permite detectar parámetros que están altamente correlacionados. Algunas relaciones serán obvias, pero otras tendremos que identificarlas para ayudar al cliente a enfocarse en ese punto de mejora que influenciará en sus objetivos principales.
\item 3.	Identificar relaciones y patrones dentro de los activos digitales: descubrir tendencias dentro de los datos nos puede dar ventaja competitiva, como detectar puntos clave que están afectando a la calidad del producto o solucionar incidencias antes de que se conviertan en mayores problemas.
\item	 4.	Desarrollar un nuevo lenguaje de negocio para contar la historia a otros: una vez que hemos descubierto nuevos insights gracias a la analítica visual, el siguiente paso es comunicarlos, ya sea con gráficos simples o visualizaciones elaboradas, pero lo importante es lograr ese engagement y transmitir el mensaje rápidamente. \\

	\begin{center}
	\includegraphics[width=10cm]{./Imagenes/imagen4} 
	\end{center}
\end{itemize} 

	%%

	%%
	%\linenumbers
	
	%% main text

\section{Ejemplo}
	

aqui el ejemplo





\section{Analisis}
	

aqui el analisis

\newpage

%%
	
	%%
	%\linenumbers
	
	%% main text
\section{Conclusion}
aqui la conclusion

%%
	
	%%
	%\linenumbers
	
	%% main text

	
	
	\newpage
	
	\bibliographystyle{apalike} 	%ESTILO
	\bibliography{BIBLIOGRAFIA}		%ARCHIVO .bib
	
	%% The Appendices part is started with the command \appendix;
	%% appendix sections are then done as normal sections
	%% \appendix
	
	%% \section{}
	%% \label{}
	
	%% References
	%%
	%% Following citation commands can be used in the body text:
	%% Usage of \cite is as follows:
	%%   \cite{key}          ==>>  [#]
	%%   \cite[chap. 2]{key} ==>>  [#, chap. 2]
	%%   \citet{key}         ==>>  Author [#]
	
	%% References with bibTeX database:
	
	
	%% Authors are advised to submit their bibtex database files. They are
	%% requested to list a bibtex style file in the manuscript if they do
	%% not want to use model1-num-names.bst.
	
	%% References without bibTeX database:
	
	% \begin{thebibliography}{00}
	
	%% \bibitem must have the following form:
	%%   \bibitem{key}...
	%%
	
	% \bibitem{}
	
	% \end{thebibliography}
	
	
\end{document}

%%
%% End of file `elsarticle-template-1-num.tex'.
